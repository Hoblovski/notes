\documentclass[a4paper]{article}

\usepackage{amsmath}
\usepackage{amsfonts}
\usepackage{amssymb}
\usepackage{xcolor}
\usepackage{listings}
\usepackage{libertine}
\setmonofont{Courier}
%\usepackage[euler-digits,euler-hat-accent]{eulervm}
\usepackage{graphicx}
\usepackage{url}

\usepackage{tabularx}
\newcolumntype{M}{>{$}l<{$}}
\newcolumntype{N}{>{$}r<{$}}
\newcolumntype{P}{>{$}c<{$}}
\newcolumntype{L}{>{\raggedright\arraybackslash}X}
\newcolumntype{R}{>{\raggedleft\arraybackslash}X}
\newcolumntype{C}{>{\centering\arraybackslash}X}
\newcolumntype{+}{>{\global\let\currentrowstyle\relax}}
\newcolumntype{^}{>{\currentrowstyle}}
\newcommand{\rowstyle}[1]{\gdef\currentrowstyle{#1}#1\ignorespaces}

\usepackage{booktabs}
\usepackage{ctex}
\setCJKmainfont[ItalicFont=Noto Sans CJK SC Bold, BoldFont=Noto Serif CJK SC Black]{Noto Serif CJK SC}

\definecolor{mygray}{rgb}{0.5,0.5,0.5}
\definecolor{backgray}{gray}{0.95}
\lstdefinestyle{somelanguage}{ %
  belowcaptionskip=1\baselineskip,
  breaklines=true,
  mathescape=true,
  showstringspaces=false,
  numbers=left,
  xleftmargin=2em,
  framexleftmargin=1.5em,
  numbersep=5pt,
  numberstyle=\tiny\color{mygray},
  basicstyle=\scriptsize\ttfamily,
  keywordstyle=\color{blue},
  commentstyle=\itshape\color{red},
  morekeywords={ },
  tabsize=2,
  backgroundcolor=\color{backgray}
}

\title{Proofs and Types}
\author{}

\begin{document}
\maketitle

\section{注意}
\begin{enumerate}
	\item 书中整数指自然数
	\item $\Rightarrow$ 指 $\to$
\end{enumerate}

\section{Chapter 1: SENSE, DENOTATION AND SEMANTICS}
\paragraph{Sense and denotation}
\url{https://www.douban.com/note/668324388/} 指示词这个词语本身是 sense,它所指示的东西是 denotation。
于是说明 $27\times 37=999$ 就是说明左右不同的 sense 的 denotation 是一样的。
\begin{quote}
given a sentence A, there are two ways of seeing it:

• as a sequence of instructions, which determine its sense \ldots

• as the ideal result found by these operations: this is its denotation.
\end{quote}

Syntax / semantics; sense / denotation; proof / truth; 的区别:传统历史上大家注重 denotation,但是计算机的兴起让人注意 sense.

\paragraph{Semantic traditions}
\textbf{Tarski 语义}:只关心句子真值(是 \textbf{t} 还是 \textbf{f})。
假设原子句子的真值已知,给出连接词的真值表,量词的真值含义,就可以定义句子的真值。

\textbf{Heyting 语义}:关心 ``what is a proof of A?'' 而非 ``when is a sentence A true?''。
Proof 自己就是一个对象,不是我们写下的东西。
我们自然知道原子句子的 proof,$A\land B$ 的 proof 是一个二元组 $(p, q)$ 其中 $p$ 是 $A$ 的 proof 而 $q$ 是 $B$ 的 proof;
$A \lor B$ 的是一个 $(i, p)$ 其中 $i\in\{0, 1\}$;
$A \to B$ 的 proof 是一个函数:$A$ 的 proof 映射成 $B$ 的 proof;
$\forall a\, .\;P$ 的 proof 是一个函数:任一论域内的的元素 $a'$ 映射成 $P[a\mapsto a']$ 的 proof。

Heyting 不是 formal system。比如,里面的 ``函数'' 就不是形式定义的。

\section{Chapter 2: NATURAL DEDUCTION}
一个证明像是一个树状结构\footnote{this view is more a graphical illusion than a mathematical reality}。

第二章自然演绎只考虑 $\to, \forall, \land$ 片段。
每个引入规则都有一个对应的消去规则。



%\bibliographystyle{plain}
%\nocite{*}
%\bibliography{TODO}
\end{document}
