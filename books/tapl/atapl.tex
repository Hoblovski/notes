\documentclass{article}

\usepackage{amsmath}
\usepackage{amsfonts}
\usepackage{amssymb}
\usepackage[inference]{semantic}
\usepackage{mathtools}

\title{Type and Programming Languages}
\author{Hoblovski}

\newcommand{\nset}{\mathbb{N}}
\newcommand{\rset}{\mathbb{R}}
\newcommand{\ite}[3]{\text{if}\; #1 \; \text{then}\; #2 \; \text{else}\; #3}
\newcommand{\lett}[3]{\text{let}\; #1\, =\, #2 \;\text{in}\; #3}
\newcommand{\lam}[2]{\lambda #1 .\;#2}
\newcommand{\lamt}[3]{\lambda #1: #2 .\;#3}
\newcommand{\lamk}[3]{\lambda #1 :: #2 .\;#3}
\newcommand{\app}[2]{#1\, #2}
\newcommand{\appp}[2]{\left(#1\, #2\right)}
\newcommand{\tapp}[2]{#1\, \left[#2\right]}
\newcommand{\typjud}[2]{\Gamma \vdash #1: #2}
\newcommand{\typjudp}[2]{\Gamma \vdash \left(#1\right): #2}
\newcommand{\typjudr}[2]{\Gamma,\; \Sigma \vdash #1: #2}
\newcommand{\typjudrp}[2]{\Gamma, \; \Sigma \vdash \left(#1\right): #2}
\newcommand{\typjudc}[3]{\Gamma \vdash #1: #2 \;| #3}
\newcommand{\typjudcp}[3]{\Gamma \vdash \left(#1\right): #2 \;| #3}
\newcommand{\mtrue}{\mathrm{true}}
\newcommand{\mfalse}{\mathrm{false}}
\newcommand{\mref}{\mathrm{ref}\, }
\newcommand{\mtop}{\mathrm{top}}
\newcommand{\mbot}{\mathrm{bot}}
\newcommand{\mRef}{\mathrm{Ref}\, }
\newcommand{\munit}{\text{unit}}
\newcommand{\dom}{\mathrm{dom}}
\newcommand{\uquant}[2]{\forall #1 \, .\; #2}
\newcommand{\equant}[2]{\exists #1 \, .\; #2}
\newcommand{\epack}[2]{#1 \,\mathrm{as}\; #2}
\newcommand{\elet}[4]{\text{let}\; \{#1, #2\}\, =\, #3 \;\text{in}\; #4}
\newcommand{\TBool}{\text{Bool}}
\newcommand{\TNat}{\text{\texttt{Nat}}}
\newcommand{\TVec}{\text{\texttt{Vec}}}
\newcommand{\lambdaLF}{\lambda_{\text{LF}}}
\newcommand{\lambdaF}{\lambda_{\text{F}}}
\newcommand{\lambdato}{\lambda_{\to}}
\newcommand{\DT}{\, .\;}

% Setting all equations left justified
%
% \documentclass[fleqn]{article}
% \usepackage{amsmath}
% \setlength{\mathindent}{2em}

\begin{document}
\maketitle
\tableofcontents

\section{Ch1: Substructural Types}
\subsection{Concepts}
\paragraph{Structural properties}
  \begin{itemize}
    \item \emph{exchange} (order of appearence in $\Gamma$ does not matter),
      \[
        \inference{
          \Gamma_1, \; x_1:T_1, x_2: T_2, \;\Gamma_2 \vdash x: T
        }{
          \Gamma_1, \; x_2:T_2, x_1: T_1, \;\Gamma_2 \vdash x: T
        }
      \]
    \item \emph{weakening} (adding extra unused assumptions)\footnote{implicit alpha conversion ;P},
      \[
        \inference{
          \Gamma_1, \Gamma_2 \vdash x:T
        }{
          \Gamma_1,\;x_1:T_1, \;\Gamma_2 \vdash x:T
        }
      \]
    \item \emph{contraction}
      \[
        \inference{
          \Gamma_1, \; x_2:T_1, x_3:T_1, \;\Gamma_2\vdash t:T_2
        }{
          \Gamma_1, \; x_1:T_1,\;\Gamma_2\vdash [x_1\mapsto x_1] [x_3\mapsto x_1]\,  t:T_2
        }
      \]
  \end{itemize}

\paragraph{Flavors}
  \begin{itemize}
    \item Linear: exchange.
      Use once.
    \item Affine: exchange + weakening
      Use at most once.
    \item Relevant: exchange + contraction
      Use at least once.
    \item Ordered: none
      Use once and use in order.
    \item Use $q$ for system (also qualifier).
      $q_1 \sqsubseteq q_2$ means $q_1$ is more restrictive (less structural properties) than $q_2$.
  \end{itemize}

\paragraph{Linear Type System}
  Use just once, so deallocate after use i.e. use = free. $\to$ no memleak, no use-after-free.
  \begin{itemize}
    \item Add qualifiers $q ::= \text{lin} \;|\; \text{un}$. Literals (bool, abs, etc.) annotated with $q$.
    \item Types are qualified pretypes $T ::= q\, P$. Pretypes are $\lambda{\to}$ types.
    \item \emph{Context split}: $\Gamma = \Gamma_1 \circ \Gamma_2$, and require any $x: \text{lin}\, T$ must appear in only one.
    \item \emph{Containment}: disallow $\text{un}$ terms to contain $\text{lin}$ terms. Subterms cannot be more restrictive.
    \item $q(T)$: $q$ is no less restrictive than $T$'s qualifier. $q(\Gamma)$: $q$ is no less restrictive than any types in $\Gamma$.
      (dzy; write like $q \sqsubseteq T$ and $q \sqsubseteq \Gamma$.)
    \item Ensure linear variable used $\ge$ once: $T-Var$.
\[
  \inference{\Gamma_1, \Gamma_2 \text{ contains no linear types}}
            {\Gamma_1, x:T, \Gamma_2\vdash x:T}
            \tag{T-Var}
\]
      example: $\text{un}~(\lambda x:\text{un}\, \TBool:\top)$ does not use $x$ and fails to type check.
    \item Ensure linear variable used $\le$ once: context split
\[
  \inference{\Gamma_1\vdash t_1: q\, T_1\to T_2 & \Gamma_2 \vdash t_2: T_1}
            {\Gamma_1 \circ \Gamma_2 \vdash t_1\, t_2:T_2}
            \tag{T-App}
\]
    \item Disallow creating less restrictive abstractions in a more restricted environment
\[
  \inference{
    q \sqsubseteq \Gamma & \Gamma, x:T_1 \vdash t: T_2
  }{
    \Gamma \vdash q\, \lambda x:T_1.t \;:\;q \, T_1\to T_2
  }
\]
    \item Algorithmic type checking: main difficulty is context split,
      e.g. $\Gamma \xrightarrow{t_1} \Gamma_{\text{remain}} \xrightarrow{t_2} \Gamma_{\text{remain}}$.
      Use notation:
\[
    \Gamma_{\text{before}} \vdash t:T ; \Gamma_{\text{after}}
\]
    \item Ensure linear variables used $\ge$ once: $\Gamma_{\text{out}} \div \Gamma_{\text{constructed variables}}$, undefined when constructed linear variables overlap with $\Gamma_{\text{out}}$ i.e. unused.
    \item Ensure linear variables used $\le$ once: trivial, $\Gamma_{\text{out}}$ chaining.
    \item Operational semantics: explicit store; remove linear variable from store when used.
  \end{itemize}

\paragraph{Others}
  \begin{itemize}
    \item TODO
  \end{itemize}

\section{Ch2: Dependent Types}
\subsection{Concepts}
  \begin{itemize}
    \item Invented by Martin Lof; inspired by Curry Howard correspondence. STLC with dependent types is denoted with $\lambdaLF$.
    \item $\lambdaLF$ introduces the \emph{dependent product type} a.k.a. \emph{pi type}, that allows term binders in types (c.f. $\lambdaF$ allows type binders).
    \item Curry Howard correspondence: $\lambdaLF$ corresponds to first-order logic.
        Types are formulae, while terms are proofs. Propositions may be mixed with types.
    \item $\prod x:T_1 \DT T_2$ is basically the good old $T_1 \to T_2$, but with a binder so $x$ may occur in $T_2$. Traditional types may be written as $\prod \_:T_1\DT T_2$.
    \item \emph{Definitional type equivalence} is crucial for $\lambdaLF$. The type checker has to know $\TVec\, 2$ is completely compatible with $\TVec\, 1+1$ \emph{at compile time}, despite syntactical distinction.
  \end{itemize}

\subsection{Elements}
  One of the simplest dependent type systems: $\lambdaLF$.

\paragraph{Terms}
  Same as $\lambdato$.

\paragraph{Types}
  \begin{align*}
    T \quad::= \quad & X \tag{type / family variable} \\
      & \prod x:T \DT K \tag{pi type} \\
      & T\, t \tag{pi type application} \\
  \end{align*}

\paragraph{Kinds}
  \textbf{No type operators}. Just to distinguish pi types and proper types.
  In fact $\prod x:T \DT K$ can be written as $T \to K$.
  There is no way to construct type families. They have to be predefined.
  \begin{align*}
      K \quad::= \quad & * \tag{proper type} \\ 
        & \prod x:T \DT K \tag{type family} \\
  \end{align*}

\paragraph{Contexts}
  \begin{align*}
    \Gamma \quad::= \quad & \emptyset \tag{empty} \\ 
      & \Gamma, x:T \tag{variable binding} \\
      & \Gamma, X :: K \tag{type variable binding}\\
  \end{align*}

\paragraph{Typing}
  With T-APP, the argument enters into the return type.
  With T-CONV, equivalent types are interchangable.
  \[
    \inference{
        x:T\in\Gamma & \Gamma\vdash T::*
    }{
        \Gamma\vdash x:T
    }
    \tag{T-VAR}
  \]
  \[
    \inference{
        \Gamma\vdash S :: * & \Gamma, x:S \vdash t:T
    }{
        \Gamma\vdash \lambda x:S\DT T\;:\;\prod x:S \DT T
    }
    \tag{T-ABS}
  \]
  \[
    \inference{
        \Gamma\vdash t_1: \prod x :S \DT T &
        \Gamma\vdash t_2: S
    }{
        \Gamma\vdash t_1\, t_2\;:\; [x\mapsto t_2] T
    }
    \tag{T-APP}
  \]
  \[
    \inference{
        \Gamma\vdash t:T_1&
        \Gamma\vdash T_1 \equiv T_2 :: *
    }{
        \Gamma\vdash t:T_2
    }
    \tag{T-CONV}
  \]

\paragraph{Kinding}
  I think there's something wrong with this figure.

\paragraph{Kind well-formedness}
  Basically it prohibits type families from being in the lhs of $\prod$s.
  Because there is no term $t$ that satisfies $t:X$ for a type family $X$.

\paragraph{Kind equivalence}
  Apart from reflexitivity, symmetry and transitivity, informally:
  \begin{itemize}
    \item Type family kinds are equivalent if the term type and the returned kinds are equivalent.
  \end{itemize}

\paragraph{Type equivalence}
  Apart from reflexitivity, symmetry and transitivity, informally
  \begin{itemize}
    \item Pi types are equivalent, if the parameter types and the return types are equivalent.
    \item Family applications are equivalent, if the families types are equivalent and the argument \emph{terms are equivalent}.
  \end{itemize}

\paragraph{Term equivalence}
  Apart from reflexitivity, symmetry and transitivity, informally:
  defined for abstraction, application, plus $\beta$-reduction and $\eta$-conversion.

  Term equivalence makes unconstrained type checking intractable.

\subsection{Examples}
\paragraph{Integer vectors}
  An integer vector with length of $n$ has type $\text{\texttt{VecInt}}\, n$.

  Thus
  \begin{itemize}
    \item \texttt{VecInt} is a \emph{family variable}; it is also a type; it has kind $\prod n:\TNat\, .\; *$.
      But note there is clear distinction between $\prod n:\TNat\, .\;*$ and $* \Rightarrow *$.
    \item $\text{\texttt{VecInt}}\, n$ is a proper type; it is a type; it has kind $*$.
    \item $\text{\texttt{append}}$ is a term; it has type $\prod n:\TNat\, .\; \text{\texttt{VecInt}}\, n \to \text{\texttt{Int}} \to \text{\texttt{VecInt}}\, n+1$.
      Thus the length invariant is encoded into the function signature.
    \item $\text{\texttt{head}}$ is a term; it has type $\prod n:\TNat\, .\;\text{\texttt{VecInt}}\, n+1$.
      It never accepts empty lists.
  \end{itemize}

\paragraph{Bounded natural numbers}
  $\text{\texttt{NatR}}\, a\, b$ represents a natural number in the range $(a, b)$, refining the integer representations.
  Then we have
  \begin{itemize}
    \item $\emptyset \quad:\quad \text{\texttt{NatR}} \quad::\quad \prod a:\TNat\, .\;\prod b: \TNat\, .\; *$
    \item $\{2, 3\} \quad:\quad \text{\texttt{NatR}}\, 2\, 3\quad::\quad *$
    \item $\lambda a: \TNat \DT \lambda b: \TNat \DT \lambda x: \text{\texttt{NatR}}\, a\, b \DT x+1$\\
        $:\quad \prod a:\TNat\DT \prod b:\TNat\DT \prod x:\text{\texttt{NatR}}\, a\, b\DT \text{\texttt{NatR}}\, a+1\, b+1$\\
        $::\quad *$
  \end{itemize}

\subsection{Takeaways}
\end{document}

